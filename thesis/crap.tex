% \label{tab:thesis_summary}
% \begin{landscape}
%     \centering
%     \captionof{table}{Thesis Summary: Research Questions, Problem Schematic and Main Results}
%     \begin{longtable}{ |p{2cm}|p{7cm} | p{7cm} | }
%      \hline 
%      Chapter & Schematic & Main Result\\ \hline
%           \chap\ \ref{chap:cim} &    \includegraphics[scale = 0.1]{figures/chap8/c43.png} \hspace{0.2cm}
%   \includegraphics[scale = 0.15]{figures/chap8/c42.png} \hspace{0.2cm}
%   \includegraphics[scale = 0.1]{figures/chap8/c41.png} \newline \tiny Can Ising Models on scale-free networks be approximated by an infinite-range homogeneous Ising model? Can this be used to investigate thermodynamic relationships between scale-free networks, lattices and cliques?   & \includegraphics[scale = 0.15]{figures/chap8/c45.pdf} \includegraphics[scale = 0.15]{figures/chap8/c44.pdf} \newline \tiny An infinite homogeneous Ising model approximation of scale-free networks shows that the connectivity of scale-free networks are between lattices and cliques \\ \hline
%      \chap\ \ref{chap:mim} & \includegraphics[scale = 0.3]{figures/chap8/c31.pdf} \newline \tiny Can Ising Models be used to simulate Healthy to Diseased Phase Transitions in Gene Regulatory Networks?   &  \includegraphics[scale = 0.2]{figures/chap8/c32.pdf} \newline \tiny Ising Model of a ferromagnetically coupled gene regulatory network shows first order phase transition\\ \hline
%      \chap\ \ref{chap:rg} &  \includegraphics[scale = 0.52]{figures/chap8/c51.pdf} \newline \tiny Can we use Renormalization methods on Scale-Free Networks to validate results obtained from Mean Field Theory?   & \includegraphics[scale = 0.12]{figures/chap8/c52.pdf} \includegraphics[scale = 0.12]{figures/chap8/c53.pdf} \newline \tiny Bond Propagation Algorithm validates Mean Field calculations \\ \hline
%      \chap\ \ref{chap:spike} &  \includegraphics[scale = 0.5]{figures/chap8/c61.pdf} \newline \tiny Is it possible to develop a method that guarantees fail-safe spike detection? & \includegraphics[scale = 0.1]{figures/chap8/c62.pdf} \newline \tiny Evolution of threshold back in time in state space ensures no spike misses \\ \hline
%      \chap\ \ref{chap:ci}  & \includegraphics[scale = 0.05]{figures/chap8/c71.jpg}  \newline \tiny Do calcium imaging data of differentiating neuronal population show signature firing patterns?   &  \includegraphics[scale = 0.1]{figures/chap8/c72.pdf} \newline \includegraphics[scale = 0.1]{figures/chap8/c73.pdf} \includegraphics[scale = 0.1]{figures/chap8/c74.pdf} \includegraphics[scale = 0.1]{figures/chap8/c75.pdf} \newline \tiny Neurons in intermediate stage of differentiation shows a transient connectivity structure and spiking behavior  \\ \hline
%      \hline
%     \end{longtable} 
% \end{landscape}



% How do local processes shape the global features of the network? How does their topology influence dynamical processes? How can we characterize the contribution of a node to a network? How can one build a simulator that can accurately simulate such large networks with non-trivial topology? 


%%%%%%%%%%%%%%%%%%%%%%%%%%%%%%%%%%%%%%%%%%%%%%%%%%%%%%%%%%%%%%%%%%%%%%%%%%%%%%%%%%%%%%%%%%%%%%%%%%%%%%%%%%%%%%%%%%%%%%%%%%%%%%%%%%%%%%%%%%%%%%%%%%%%%%%%%%
